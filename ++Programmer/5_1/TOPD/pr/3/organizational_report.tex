\documentclass[14pt,a4paper]{extarticle}
\usepackage[T2A]{fontenc}
\usepackage[utf8]{inputenc}
\usepackage[english,russian]{babel}
\usepackage{lmodern}
\usepackage{geometry}
\usepackage{array}
\usepackage{tabularx}
\usepackage{longtable}
\usepackage{ltablex}
\usepackage{booktabs}
\usepackage{hyperref}
\usepackage{indentfirst}
\usepackage{microtype}
\usepackage{graphicx}
\usepackage{setspace}
\usepackage{mathptmx}

\keepXColumns
\setlength{\parindent}{3em}
\geometry{left=25mm,right=25mm,top=25mm,bottom=25mm}

\title{Отчёт по организационно-правовым формам}
\author{Выполнил студент группы М-РИТ-25, Станиславчук Сергей}

\begin{document}

\setstretch{1.5}

\noindent
\begin{minipage}[t]{0.01\textwidth}
    \includegraphics{LSTU.jpg}
\end{minipage}%
\begin{minipage}[c]{1.1\linewidth}
    \centering
    \textbf{МИНИСТЕРСТВО НАУКИ И ВЫСШЕГО ОБРАЗОВАНИЯ РФ} \\[1em]
    \textbf{ФЕДЕРАЛЬНОЕ ГОСУДАРСТВЕННОЕ БЮДЖЕТНОЕ} \\
    \textbf{ОБРАЗОВАТЕЛЬНОЕ УЧРЕЖДЕНИЕ ВЫСШЕГО ОБРАЗОВАНИЯ} \\
    \textbf{«ЛИПЕЦКИЙ ГОСУДАРСТВЕННЫЙ ТЕХНИЧЕСКИЙ УНИВЕРСИТЕТ»}
\end{minipage}%

\vspace{3em}

\noindent
\hspace{1em}%
\begin{minipage}[t]{0.8\textwidth}
\textbf{Институт} \hspace{2em} компьютерных наук \\
\textbf{Кафедра} \hspace{2em} автоматизированных систем управления
\end{minipage}


\vspace{6em}

\begin{center}
\textbf{\Large ПРАКТИЧЕСКОЕ ЗАДАНИЕ №3} \\
по дисциплине "Основы предпринимательства"
\end{center}

\vspace{4em}

\noindent
\begin{tabular*}{\textwidth}{@{\extracolsep{\fill}} l c r }
    Студент М-РИТ-25-1 & \makebox[4cm]{\hrulefill} & Станиславчук С.М. \\[2em]
Руководитель & \\
Канд. юр. наук, доцент & \makebox[4cm]{\hrulefill} & Моисеева И.И. \\
\end{tabular*}

\vfill
\begin{center}
    Липецк, 2025
\end{center}

\thispagestyle{empty}

\newpage
\tableofcontents
\newpage

\section{Раздел: Описания кейсов}
\subsection*{Кейс 1}
\begin{sloppypar}
Для сети из десяти кофеен, основанной двумя партнёрами с уставным капиталом 100
миллионов рублей, наиболее целесообразной организационно-правовой формой
является общество с ограниченной ответственностью (ООО). Такая форма позволяет
распределить доли участия между партнёрами в соответствии с их вкладами, чётко
закрепив права и обязанности каждого в учредительных документах. ООО
обеспечивает ограниченную ответственность участников, что защищает их личное
имущество от рисков бизнеса, и при этом даёт возможность привлечения сторонних
инвесторов в будущем за счёт продажи долей или увеличения уставного капитала.
Кроме того, эта форма является оптимальной для предприятий среднего уровня,
совмещая гибкость управления и прозрачность для потенциальных инвесторов.
\end{sloppypar}

\subsection*{Кейс 2}
В случае с одиночным предпринимателем, создающим и продающим изделия ручной
работы через интернет, оптимальным решением будет регистрация в качестве
индивидуального предпринимателя (ИП) или самозанятого. Такая форма обеспечивает
простоту регистрации, минимальные издержки на ведение бухгалтерского учёта и
упрощённую систему налогообложения. Для мастера, работающего самостоятельно,
нет необходимости в сложной структуре управления или уставном капитале. При
небольших объёмах дохода наиболее выгодным является режим налога на
профессиональный доход (НПД), позволяющий легально получать прибыль с
минимальной налоговой нагрузкой. Если же бизнес со временем расширится,
предприниматель может легко перейти на форму ИП с УСН, сохранив контроль над
деятельностью.

\subsection*{Кейс 3}
Для крупного производства мебели с десятками сотрудников, значительным уставным
капиталом и планами выхода на международный рынок оптимальной
организационно-правовой формой является акционерное общество (АО),
предпочтительно публичное (ПАО), если предполагается привлечение капитала через
продажу акций. Данная форма наиболее отвечает потребностям масштабного бизнеса,
обеспечивая возможность привлечения крупных инвестиций, участия иностранных
партнёров и выхода на фондовый рынок. Акционерная структура позволяет чётко
разграничить права собственников и менеджмента, а также повышает прозрачность
компании для инвесторов. Кроме того, ПАО даёт гибкость в управлении активами и
возможность дальнейшего расширения без кардинальных изменений в структуре
собственности.

\newpage
\section{Раздел: Сравнительная таблица основных ОПФ}
\begin{tabularx} {\textwidth} {>{\bfseries}X X X X}
\toprule
Организационно-правовая форма & Основные особенности & Преимущества & Недостатки \\
\midrule
\endhead
ИП & Один владелец, простая регистрация,
    упрощённая отчётность & Минимальные налоги, быстрое открытие и закрытие,
    полное распоряжение прибылью & Личная ответственность по обязательствам,
    ограниченные возможности привлечения инвестиций \\
\addlinespace
Самозанятый & Физическое лицо без статуса ИП, работающее на себя & Самый
    низкий налог (4--6\%), отсутствие бухгалтерии, регистрация через приложение
    & Нельзя нанимать сотрудников, ограничение по годовому доходу \\
\addlinespace
ООО & Один или несколько участников,
    наличие уставного капитала & Ограниченная ответственность, возможность
    привлечения инвесторов, чёткое распределение долей & Более сложная
    отчётность, необходимость ведения бухгалтерии \\
\addlinespace
АО & Крупная компания с выпуском акций (ПАО или АО) &
    Возможность привлечения значительных инвестиций, участие множества
    акционеров & Высокие затраты на регистрацию и ведение отчётности, сложная
    структура управления \\
\bottomrule
\end{tabularx}

\newpage
\section{Раздел: Рекомендация начинающему предпринимателю}
Если рассматривать себя начинающим предпринимателем, планирующим открыть
небольшой бизнес~--- например, мастерскую по ремонту техники,~--- наиболее
подходящей организационно-правовой формой будет индивидуальное
предпринимательство (ИП). Такая форма оптимальна для малого бизнеса, где
деятельность ведётся одним человеком или небольшой командой, а стартовые
вложения ограничены. Регистрация ИП проста и требует минимальных затрат,
налоговая отчётность несложна, а прибыль поступает напрямую владельцу. При этом
ИП позволяет нанимать сотрудников, расширять бизнес и при необходимости
переходить на другие системы налогообложения. Основным риском остаётся личная
имущественная ответственность по обязательствам, однако при небольших объёмах
деятельности этот риск приемлем. Альтернативой могла бы стать самозанятость, но
она подходит лишь для работы без сотрудников и при небольшом обороте, поэтому
для мастерской ИП является более гибким и перспективным вариантом.

\newpage
\section{Раздел: Сравнительный анализ (таблицы по парам)}
\subsection*{1) ИП vs ООО}
\begin{tabularx}{\textwidth}{X X X}
\toprule
Критерий & ИП & ООО \\
\midrule
Количество учредителей & 1 человек & От 1 до 50 участников \\
Минимальный уставный капитал & Не требуется & Минимум 10 000~руб. \\
Порядок регистрации & Простая процедура, требуется заявление, паспорт и квитанция об оплате госпошлины, срок --- 3--5 дней & Более сложная регистрация, необходим устав, решение/договор об учреждении, сведения об участниках, срок --- 5--7 дней \\
Ответственность учредителей & Несёт полную имущественную ответственность & Участники отвечают только в пределах своих вкладов \\
Управление & Управляет сам предприниматель & Руководство может осуществляться директором, участники принимают решения на собрании \\
Налогообложение & УСН, ПСН, ОСНО, НПД (в зависимости от деятельности) & УСН, ОСНО, иногда ЕСХН \\
Репутация и доверие & Средний уровень доверия, больше подходит для малого бизнеса & Более высокий уровень доверия со стороны банков и инвесторов \\
Возможность привлечения инвестиций & Практически отсутствует & Возможна продажа долей и привлечение новых участников \\
\bottomrule
\end{tabularx}

\newpage
\subsection*{2) ООО vs АО}
\begin{tabularx}{\textwidth}{X X X}
\toprule
Критерий & ООО & АО (ПАО/НАО) \\
\midrule
Количество учредителей & От 1 до 50 & От 1 до неограниченного числа акционеров \\
Минимальный уставный капитал & 10 000~руб. & 100 000~руб. для ПАО, 10 000~руб. для НАО \\
Порядок регистрации & Стандартная процедура, регистрация устава, учредительного договора, 5--7 дней & Более сложная, требует регистрации выпуска акций, раскрытия информации, срок дольше \\
Ответственность учредителей & В пределах вклада в уставный капитал & В пределах стоимости акций \\
Управление & Общее собрание участников, директор & Общее собрание акционеров, совет директоров, исполнительный орган \\
Налогообложение & УСН или ОСНО & Обычно ОСНО, сложнее отчётность \\
Репутация и доверие & Высокая, но ограничена масштабом бизнеса & Очень высокая, подходит для крупных компаний \\
Возможность привлечения инвестиций & Через приём новых участников или увеличение капитала & Через выпуск и продажу акций, проще привлечь крупных инвесторов \\
\bottomrule
\end{tabularx}

\newpage
\subsection*{3) ИП vs Самозанятый}
\begin{tabularx}{\textwidth}{X X X}
\toprule
Критерий & ИП & Самозанятый \\
\midrule
Количество учредителей & 1 человек & 1 человек \\
Минимальный уставный капитал & Не требуется & Не требуется \\
Порядок регистрации & Регистрация через налоговую инспекцию, госпошлина 800~руб., срок 3--5 дней & Регистрация через приложение «Мой налог» за несколько минут, без госпошлины \\
Ответственность учредителей & Полная имущественная ответственность & Полная имущественная ответственность \\
Управление & Самостоятельное & Самостоятельное \\
Налогообложение & УСН, ПСН, ОСНО & Налог на профессиональный доход: 4\% (физлица) и 6\% (юрлица) \\
Репутация и доверие & Выше, чем у самозанятых, возможна работа с юрлицами & Ограниченное доверие, часто подходит только для частных заказов \\
Возможность привлечения инвестиций & Невелика, но можно нанимать сотрудников и развивать бизнес & Отсутствует, нельзя нанимать работников и расширять бизнес \\
\bottomrule
\end{tabularx}

\newpage
\section{Раздел: Кейс-стади: выбор ОПФ}
\subsection*{Кейс 1}
\textbf{Ваш выбор:} Общество с ограниченной ответственностью (ООО).

\textbf{Обоснование:} Два партнёра, Иван и Пётр, планируют совместный бизнес с равным участием в капитале и управлении. ООО идеально подходит для такой ситуации, поскольку позволяет распределить доли между учредителями и закрепить равные права при принятии решений через устав и учредительный договор. При этом личная ответственность партнёров ограничена их вкладами, что снижает риски. ООО также даёт возможность в будущем привлекать инвесторов путём увеличения уставного капитала или продажи долей.

\textbf{Потенциальные преимущества:} Ограниченная ответственность участников, чёткое распределение долей, возможность привлечения инвестиций, репутация юридического лица, гибкость в управлении.

\textbf{Потенциальные недостатки:} Более сложная регистрация и отчётность по сравнению с ИП, обязательное ведение бухгалтерии, необходимость официального оформления внутренних решений (например, собраний участников).

\subsection*{Кейс 2}
\textbf{Ваш выбор:} Акционерное общество (АО), на начальном этапе возможно создание общества с ограниченной ответственностью с последующим преобразованием в АО.

\textbf{Обоснование:} Анна развивает проект с потенциалом масштабирования и привлечения инвестиций, включая возможность выпуска акций в будущем. Для старта, когда уставный капитал небольшой, можно начать с ООО, а по мере роста и необходимости привлечения крупных инвесторов преобразовать бизнес в АО. Эта форма позволит выпускать акции, распределять доли между участниками команды и привлечёнными инвесторами, а также повысит доверие со стороны крупных клиентов.

\textbf{Потенциальные преимущества:} Возможность выпуска акций, прозрачная структура управления, высокий уровень доверия со стороны инвесторов и партнёров, перспективы выхода на рынок капитала.

\textbf{Потенциальные недостатки:} Более сложная регистрация и отчётность, необходимость соблюдения корпоративных процедур, высокие затраты на ведение бухгалтерии и раскрытие информации.

\subsection*{Кейс 3}
\textbf{Ваш выбор:} Общество с ограниченной ответственностью (ООО) или непубличное акционерное общество (НАО), в зависимости от числа инвесторов и их роли.

\textbf{Обоснование:} Сергей планирует производство с привлечением нескольких инвесторов и хочет ограничить личную ответственность. ООО подойдёт, если участников немного и планируется активное участие в управлении. Если же число инвесторов больше и они заинтересованы прежде всего в вложениях, а не в управлении, оптимальнее будет НАО, которое позволяет распределять доли в виде акций без публичного размещения. Обе формы обеспечивают ограниченную ответственность и удобны для совместного бизнеса.

\textbf{Потенциальные преимущества:} Ограничение личной ответственности, возможность привлечения инвестиций, гибкое распределение долей (или акций), юридическая прозрачность.

\textbf{Потенциальные недостатки:} Более высокая административная нагрузка по сравнению с ИП, необходимость ведения бухгалтерского учёта, оформление корпоративных решений и возможные конфликты интересов между инвесторами.

\newpage
\section{Раздел: Информационный анализ по ООО}
\subsection*{Правовая база}
Деятельность и порядок существования ООО в РФ регулируются специальным федеральным законом «Об обществах с ограниченной ответственностью» (№ 14-ФЗ) в связке с положениями Гражданского кодекса и общими нормами, регулирующими юридические лица; эти акты формируют корпоративное законодательство, требования к уставному капиталу, процедурам принятия решений и защите прав кредиторов и участников.

\subsection*{Этапы регистрации}
Подготовьте учредительные документы (устав; при нескольких учредителях --- договор об учреждении или протокол собрания), заполните форму заявления на госрегистрацию Р11001, оплатите госпошлину и приложите квитанцию, подготовьте решение/протокол о создании и сведения об адресе и руководителе; подача документов --- в территориальную инспекцию ФНС по месту нахождения будущего общества (можно онлайн через личный кабинет или МФЦ). При корректно оформленных документах регистрация по закону занимает \textbf{обычно 3 рабочих дня} с момента поступления документов в ФНС; результат --- запись в ЕГРЮЛ и выписка. Обратите внимание на обязательные последующие шаги: постановка на учёт в фондах, получение сообщений о внесении в реестр и прочее.

\subsection*{Минимальный уставный капитал, оплата и сроки}
Минимальный размер уставного капитала для ООО --- \textbf{10 000 рублей}; при внесении капитала чаще всего часть либо весь капитал вносят деньгами на расчётный счёт или в кассу до регистрации в зависимости от договорённостей учредителей; если уставный капитал превышает минимальный предел, оплата частично неденежным имуществом потребует оценки и оформления.

\subsection*{Налогообложение}
ООО не ограничено одним режимом: основной (ОСНО) --- общий режим с НДС, налогом на прибыль и прочими обязательными платежами; распространённый для малых и средних компаний --- упрощённая система (УСН) с объектом «доходы» или «доходы минус расходы» при соответствующем пороге выручки; для сельхозпроизводителей --- ЕСХН; существуют и другие специальные варианты (агрегированные и т.п.), но ООО не применяет режим НПД или патент, предназначенные для физических лиц/ИП. Помимо непосредственно налогов, ООО уплачивает страховые взносы за сотрудников и соблюдает расчёты по НДФЛ при выплатах работникам.

\subsection*{Преимущества и недостатки}
Преимущества: ограниченная ответственность; гибкость в управлении и структуре владения; возможности для роста через привлечение инвестиций; доверие рынка. \\
Недостатки: административная нагрузка; стоимость создания; корпоративные риски при недостаточной формализации; налоговые обязательства при работе на ОСНО.

\newpage
\section{Тест}

\begin{enumerate}
    \item а) общей совместной собственностью участников
    \item в) с согласия остальных участников
    \item в) всем своим имуществом в одинаковом размере, кратном к стоимости их вкладов
    \item в) ограниченную ответственность в пределах пая
    \item а) ему должна быть выплачена стоимость части имущества, соответствующая его доле в уставном капитале общества
    \item в) сохраняют, если это предусмотрено в учредительном договоре и уставе объединения
    \item б) другому члену кооператива
    \item б) индивидуальные предприниматели и (или) коммерческие организации
    \item а) граждане
    \item в) участвовать в собрании акционеров с правом голоса по всем вопросам его компетенции, на получение дивидендов, а также на часть имущества в случае его ликвидации
    \item в) резервный капитал (фонд)
    \item в) общее собрание акционеров
    \item а) да
    \item а) да
    \item б) собрание вкладчиков
    \item А) цель получения прибыли
    \item в) общество с ограниченной ответственностью (ООО)
    \item б) при ведении консалтингового бизнеса
    \item б) участники рискуют только в пределах своих вкладов в уставный капитал
    \item в) акционерное общество (АО)
\end{enumerate}

\end{document}
